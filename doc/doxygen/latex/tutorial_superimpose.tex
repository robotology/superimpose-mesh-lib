This tutorial introduces the basic steps to superimpose an object, i.\+e. a mesh, on top of an image by using the \mbox{\hyperlink{classSICAD}{S\+I\+C\+AD}} class and its methods.~\newline


Just have a look at the following commented, ready-\/to-\/go, code snippets!~\newline


\subsection*{Default easy-\/peasy \mbox{\hyperlink{classSICAD}{S\+I\+C\+AD}}}





\href{https://github.com/robotology/superimpose-mesh-lib/blob/master/doc/tutorial_code/tutorial_superimpose.cpp}{\tt View it online.} 
\begin{DoxyCodeInclude}
1 \textcolor{preprocessor}{#include <\mbox{\hyperlink{SICAD_8h}{SuperimposeMesh/SICAD.h}}>}
2 
3 \textcolor{preprocessor}{#include <cmath>}
4 \textcolor{preprocessor}{#include <exception>}
5 \textcolor{preprocessor}{#include <iostream>}
6 
7 \textcolor{preprocessor}{#include <glm/gtc/matrix\_transform.hpp>}
8 
9 \textcolor{preprocessor}{#include <opencv2/core/core.hpp>}
10 \textcolor{preprocessor}{#include <opencv2/imgcodecs/imgcodecs.hpp>}
11 \textcolor{preprocessor}{#include <opencv2/imgproc/imgproc.hpp>}
12 
13 
14 \textcolor{keywordtype}{int} \mbox{\hyperlink{tutorial__superimpose_8cpp_ae66f6b31b5ad750f1fe042a706a4e3d4}{main}}()
15 \{\textcolor{comment}{}
16 \textcolor{comment}{    /**}
17 \textcolor{comment}{     * We want to rendere a single mesh on 1 OpenGL viewport.}
18 \textcolor{comment}{     * The SICAD class can be profitably used to accomplish this!}
19 \textcolor{comment}{     *}
20 \textcolor{comment}{     * We first need some parameters to properly draw the object.}
21 \textcolor{comment}{     * These parameters are the hypothetical intrinsic camera parameters that}
22 \textcolor{comment}{     * would be used, or that is actually used, to take grab the image on which}
23 \textcolor{comment}{     * we want to superimpose the mesh.}
24 \textcolor{comment}{     *}
25 \textcolor{comment}{     * For example, we suppose to have a 320x240 camera with focal length}
26 \textcolor{comment}{     * of 257.34 pixels and exact camera center (principal point) at (160, 120).}
27 \textcolor{comment}{     **/}
28     \textcolor{keyword}{const} \textcolor{keywordtype}{unsigned} \textcolor{keywordtype}{int} cam\_width  = 320;
29     \textcolor{keyword}{const} \textcolor{keywordtype}{unsigned} \textcolor{keywordtype}{int} cam\_height = 240;
30     \textcolor{keyword}{const} \textcolor{keywordtype}{float}        cam\_fx     = 257.34;
31     \textcolor{keyword}{const} \textcolor{keywordtype}{float}        cam\_cx     = 160;
32     \textcolor{keyword}{const} \textcolor{keywordtype}{float}        cam\_fy     = 257.34;
33     \textcolor{keyword}{const} \textcolor{keywordtype}{float}        cam\_cy     = 120;
34 
35 \textcolor{comment}{}
36 \textcolor{comment}{    /**}
37 \textcolor{comment}{     * Next, we need a mesh file!}
38 \textcolor{comment}{     * For example, we may want to superimpose the good ol' fiend of Space}
39 \textcolor{comment}{     * Invader arcade game.}
40 \textcolor{comment}{     * Here, by chance, we have a mesh of it ready to for you!}
41 \textcolor{comment}{     *}
42 \textcolor{comment}{     * To associate a mesh to a particular object the SICAD::ModelPathContainer}
43 \textcolor{comment}{     * comes into play. In this way we can associate a tag to a mesh model and}
44 \textcolor{comment}{     * use it during rendering to assign to it a particular pose (position and}
45 \textcolor{comment}{     * orientation).}
46 \textcolor{comment}{     *}
47 \textcolor{comment}{     * NOTE: supported mesh format are the one provided by the ASSIMP library.}
48 \textcolor{comment}{     *       Have a look here: http://assimp.org/main\_features\_formats.html}
49 \textcolor{comment}{     **/}
50     \mbox{\hyperlink{classSICAD_a9e1e1460d4c0f331b4fd015aae4dd721}{SICAD::ModelPathContainer}} obj;
51     obj.emplace(\textcolor{stringliteral}{"alien"}, \textcolor{stringliteral}{"./spaceinvader.obj"});
52 
53 \textcolor{comment}{}
54 \textcolor{comment}{    /**}
55 \textcolor{comment}{     * We create a SICAD object by just passing all the camera parameters.}
56 \textcolor{comment}{     **/}
57     \mbox{\hyperlink{classSICAD}{SICAD}} si\_cad(obj, cam\_width, cam\_height, cam\_fx, cam\_fy, cam\_cx, cam\_cy);
58 
59 \textcolor{comment}{}
60 \textcolor{comment}{    /**}
61 \textcolor{comment}{     * We are close to getting our superimposed mesh!}
62 \textcolor{comment}{     * We now need to choose a pose (position and orientation) to which we}
63 \textcolor{comment}{     * render the Space Invader fiend.}
64 \textcolor{comment}{     * Suppose we want it right in front of us, 10 cm away from the camera.}
65 \textcolor{comment}{     * Note that the OpenGL convention is right handed, expressed in meters,}
66 \textcolor{comment}{     * with the z axis coming out from the screen.}
67 \textcolor{comment}{     * This imply that we want the alien at -0.10 m on the z axis.}
68 \textcolor{comment}{     * As far as the orientation is concerned we want it facing us.}
69 \textcolor{comment}{     * Since we have to pass an axis-angle vector and that all zeros is invalid,}
70 \textcolor{comment}{     * we can just pass any versor with a 0 degree angle.}
71 \textcolor{comment}{     *}
72 \textcolor{comment}{     * We now have to associate the pose with a particular mesh model.}
73 \textcolor{comment}{     * Do you remember that we used "alien" for the Space Invader fiend?}
74 \textcolor{comment}{     * Then we just have to associate a tag to the pose and we are ready to}
75 \textcolor{comment}{     * render!}
76 \textcolor{comment}{     * To do so, we just use Superimpose::ModelPose and}
77 \textcolor{comment}{     * Superimpose::ModelPoseContainer as follows.}
78 \textcolor{comment}{     **/}
79     \mbox{\hyperlink{classSuperimpose_a85d40a5caf19f486d1e0c15c0a025378}{Superimpose::ModelPose}} obj\_pose(7);
80     obj\_pose[0] = 0;
81     obj\_pose[1] = 0;
82     obj\_pose[2] = -0.1;
83     obj\_pose[3] = 1;
84     obj\_pose[4] = 0;
85     obj\_pose[5] = 0;
86     obj\_pose[6] = 0;
87 
88     \mbox{\hyperlink{classSuperimpose_a178e3d4e2def6635bfcf9454dd4b5d22}{Superimpose::ModelPoseContainer}} objpose\_map;
89     objpose\_map.emplace(\textcolor{stringliteral}{"alien"}, obj\_pose);
90 
91 \textcolor{comment}{}
92 \textcolor{comment}{    /**}
93 \textcolor{comment}{     * Finally we trivially set the pose of the camera as follows.}
94 \textcolor{comment}{     *}
95 \textcolor{comment}{     * Q: why don't we have another Superimpose:: object to do this?}
96 \textcolor{comment}{     * W: well...it's under development!}
97 \textcolor{comment}{     **/}
98     \textcolor{keywordtype}{double} cam\_x[] = \{  0, 0, 0\};
99     \textcolor{keywordtype}{double} cam\_o[] = \{1.0, 0, 0, 0\};
100 
101 \textcolor{comment}{}
102 \textcolor{comment}{    /**}
103 \textcolor{comment}{     * It's render time!}
104 \textcolor{comment}{     * We save the output of the render right into a cv::Mat and we can use}
105 \textcolor{comment}{     * the well known OpenCV facilities to do whatever we want with it, for}
106 \textcolor{comment}{     * example write it on the filesystem.}
107 \textcolor{comment}{     **/}
108     cv::Mat img;
109     si\_cad.superimpose(objpose\_map, cam\_x, cam\_o, img);
110     cv::imwrite(\textcolor{stringliteral}{"./spaceinvader.jpg"}, img);
111 
112     \textcolor{keywordflow}{return} EXIT\_SUCCESS;
113 \}
\end{DoxyCodeInclude}


If instead you want to provide your own Open\+GL shader files the code changes just a tiny bit\+: you just have to provide two more paramenters to the the \mbox{\hyperlink{classSICAD}{S\+I\+C\+AD}} constructor.

\subsection*{Use your own Open\+GL shaders in \mbox{\hyperlink{classSICAD}{S\+I\+C\+AD}}}





\href{https://github.com/robotology/superimpose-mesh-lib/blob/master/doc/tutorial_code/tutorial_superimpose_customshader.cpp}{\tt View it online.} 
\begin{DoxyCodeInclude}
1 \textcolor{preprocessor}{#include <\mbox{\hyperlink{SICAD_8h}{SuperimposeMesh/SICAD.h}}>}
2 
3 \textcolor{preprocessor}{#include <cmath>}
4 \textcolor{preprocessor}{#include <exception>}
5 \textcolor{preprocessor}{#include <iostream>}
6 
7 \textcolor{preprocessor}{#include <glm/gtc/matrix\_transform.hpp>}
8 
9 \textcolor{preprocessor}{#include <opencv2/core/core.hpp>}
10 \textcolor{preprocessor}{#include <opencv2/imgcodecs/imgcodecs.hpp>}
11 \textcolor{preprocessor}{#include <opencv2/imgproc/imgproc.hpp>}
12 
13 
14 \textcolor{keywordtype}{int} \mbox{\hyperlink{tutorial__superimpose__customshader_8cpp_ae66f6b31b5ad750f1fe042a706a4e3d4}{main}}()
15 \{\textcolor{comment}{}
16 \textcolor{comment}{    /**}
17 \textcolor{comment}{     * We want to rendere a single mesh on 1 OpenGL viewport.}
18 \textcolor{comment}{     * The SICAD class can be profitably used to accomplish this!}
19 \textcolor{comment}{     *}
20 \textcolor{comment}{     * We first need some parameters to properly draw the object.}
21 \textcolor{comment}{     * These parameters are the hypothetical intrinsic camera parameters that}
22 \textcolor{comment}{     * would be used, or that is actually used, to take grab the image on which}
23 \textcolor{comment}{     * we want to superimpose the mesh.}
24 \textcolor{comment}{     *}
25 \textcolor{comment}{     * For example, we suppose to have a 320x240 camera with focal length}
26 \textcolor{comment}{     * of 257.34 pixels and exact camera center (principal point) at (160, 120).}
27 \textcolor{comment}{     **/}
28     \textcolor{keyword}{const} \textcolor{keywordtype}{unsigned} \textcolor{keywordtype}{int} cam\_width  = 320;
29     \textcolor{keyword}{const} \textcolor{keywordtype}{unsigned} \textcolor{keywordtype}{int} cam\_height = 240;
30     \textcolor{keyword}{const} \textcolor{keywordtype}{float}        cam\_fx     = 257.34;
31     \textcolor{keyword}{const} \textcolor{keywordtype}{float}        cam\_cx     = 160;
32     \textcolor{keyword}{const} \textcolor{keywordtype}{float}        cam\_fy     = 257.34;
33     \textcolor{keyword}{const} \textcolor{keywordtype}{float}        cam\_cy     = 120;
34 
35 \textcolor{comment}{}
36 \textcolor{comment}{    /**}
37 \textcolor{comment}{     * Next, we need a mesh file!}
38 \textcolor{comment}{     * For example, we may want to superimpose the good ol' fiend of Space}
39 \textcolor{comment}{     * Invader arcade game.}
40 \textcolor{comment}{     * Here, by chance, we have a mesh of it ready to for you!}
41 \textcolor{comment}{     *}
42 \textcolor{comment}{     * To associate a mesh to a particular object the SICAD::ModelPathContainer}
43 \textcolor{comment}{     * comes into play. In this way we can associate a tag to a mesh model and}
44 \textcolor{comment}{     * use it during rendering to assign to it a particular pose (position and}
45 \textcolor{comment}{     * orientation).}
46 \textcolor{comment}{     *}
47 \textcolor{comment}{     * NOTE: supported mesh format are the one provided by the ASSIMP library.}
48 \textcolor{comment}{     *       Have a look here: http://assimp.org/main\_features\_formats.html}
49 \textcolor{comment}{     **/}
50     \mbox{\hyperlink{classSICAD_a9e1e1460d4c0f331b4fd015aae4dd721}{SICAD::ModelPathContainer}} obj;
51     obj.emplace(\textcolor{stringliteral}{"alien"}, \textcolor{stringliteral}{"./spaceinvader.obj"});
52 
53 \textcolor{comment}{}
54 \textcolor{comment}{    /**}
55 \textcolor{comment}{     * We create a SICAD object by passing all the camera parameters, the shader}
56 \textcolor{comment}{     * folder path containing the shader code (don't worry, we provide basic}
57 \textcolor{comment}{     * shaders as well!).}
58 \textcolor{comment}{     * Note that for this simple tutorial we assume that the shader files are}
59 \textcolor{comment}{     * (magically) in the same folder of the executable, that's why we use ".".}
60 \textcolor{comment}{     *}
61 \textcolor{comment}{     * Shader code is included at the end of this snippet code. Have a look at}
62 \textcolor{comment}{     * it, it's really simple! If you are not used to it, there are plenty of}
63 \textcolor{comment}{     * tutorials online for writing OpenGL shaders.}
64 \textcolor{comment}{     **/}
65     \mbox{\hyperlink{classSICAD}{SICAD}} si\_cad(obj, cam\_width, cam\_height, cam\_fx, cam\_fy, cam\_cx, cam\_cy, 1, \textcolor{stringliteral}{"."});
66 
67 \textcolor{comment}{}
68 \textcolor{comment}{    /**}
69 \textcolor{comment}{     * We are close to getting our superimposed mesh!}
70 \textcolor{comment}{     * We now need to choose a pose (position and orientation) to which we}
71 \textcolor{comment}{     * render the Space Invader fiend.}
72 \textcolor{comment}{     * Suppose we want it right in front of us, 10 cm away from the camera.}
73 \textcolor{comment}{     * Note that the OpenGL convention is right handed, expressed in meters,}
74 \textcolor{comment}{     * with the z axis coming out from the screen.}
75 \textcolor{comment}{     * This imply that we want the alien at -0.10 m on the z axis.}
76 \textcolor{comment}{     * As far as the orientation is concerned we want it facing us.}
77 \textcolor{comment}{     * Since we have to pass an axis-angle vector and that all zeros is invalid,}
78 \textcolor{comment}{     * we can just pass any versor with a 0 degree angle.}
79 \textcolor{comment}{     *}
80 \textcolor{comment}{     * We now have to associate the pose with a particular mesh model.}
81 \textcolor{comment}{     * Do you remember that we used "alien" for the Space Invader fiend?}
82 \textcolor{comment}{     * Then we just have to associate a tag to the pose and we are ready to}
83 \textcolor{comment}{     * render!}
84 \textcolor{comment}{     * To do so, we just use Superimpose::ModelPose and}
85 \textcolor{comment}{     * Superimpose::ModelPoseContainer as follows.}
86 \textcolor{comment}{     **/}
87     \mbox{\hyperlink{classSuperimpose_a85d40a5caf19f486d1e0c15c0a025378}{Superimpose::ModelPose}} obj\_pose(7);
88     obj\_pose[0] = 0;
89     obj\_pose[1] = 0;
90     obj\_pose[2] = -0.1;
91     obj\_pose[3] = 1;
92     obj\_pose[4] = 0;
93     obj\_pose[5] = 0;
94     obj\_pose[6] = 0;
95 
96     \mbox{\hyperlink{classSuperimpose_a178e3d4e2def6635bfcf9454dd4b5d22}{Superimpose::ModelPoseContainer}} objpose\_map;
97     objpose\_map.emplace(\textcolor{stringliteral}{"alien"}, obj\_pose);
98 
99 \textcolor{comment}{}
100 \textcolor{comment}{    /**}
101 \textcolor{comment}{     * Finally we trivially set the pose of the camera as follows.}
102 \textcolor{comment}{     *}
103 \textcolor{comment}{     * Q: why don't we have another Superimpose:: object to do this?}
104 \textcolor{comment}{     * W: well...it's under development!}
105 \textcolor{comment}{     **/}
106     \textcolor{keywordtype}{double} cam\_x[] = \{  0, 0, 0\};
107     \textcolor{keywordtype}{double} cam\_o[] = \{1.0, 0, 0, 0\};
108 
109 \textcolor{comment}{}
110 \textcolor{comment}{    /**}
111 \textcolor{comment}{     * It's render time!}
112 \textcolor{comment}{     * We save the output of the render right into a cv::Mat and we can use}
113 \textcolor{comment}{     * the well known OpenCV facilities to do whatever we want with it, for}
114 \textcolor{comment}{     * example write it on the filesystem.}
115 \textcolor{comment}{     **/}
116     cv::Mat img;
117     si\_cad.superimpose(objpose\_map, cam\_x, cam\_o, img);
118     cv::imwrite(\textcolor{stringliteral}{"./spaceinvader.jpg"}, img);
119 
120     \textcolor{keywordflow}{return} EXIT\_SUCCESS;
121 \}
\end{DoxyCodeInclude}


Here are the shaders and the Space Invaders mesh model\+:
\begin{DoxyItemize}
\item \href{https://github.com/robotology/superimpose-mesh-lib/blob/master/doc/tutorial_code/shader_model.vert}{\tt Model vertex shader}
\item \href{https://github.com/robotology/superimpose-mesh-lib/blob/master/doc/tutorial_code/shader_model_texture.vert}{\tt Textured model vertex shader}
\item \href{https://github.com/robotology/superimpose-mesh-lib/blob/master/doc/tutorial_code/shader_model.frag}{\tt Model fragment shader}
\item \href{https://github.com/robotology/superimpose-mesh-lib/blob/master/doc/tutorial_code/shader_background.vert}{\tt Background vertex shader}
\item \href{https://github.com/robotology/superimpose-mesh-lib/blob/master/doc/tutorial_code/shader_background.frag}{\tt Background fragment shader}
\item \href{https://github.com/robotology/superimpose-mesh-lib/blob/master/doc/tutorial_code/spaceinvader.obj}{\tt Space Invader fiend}
\end{DoxyItemize}

⚠️ The four shaders must have the following {\bfseries exact names}\+:
\begin{DoxyItemize}
\item {\itshape shader\+\_\+model.\+vert} for the model vertex shader
\item {\itshape shader\+\_\+model\+\_\+texture.\+vert} for the textured model vertex shader
\item {\itshape shader\+\_\+model.\+frag} for the model fragment shader
\item {\itshape shader\+\_\+background.\+vert} for the background vertex shader
\item {\itshape shader\+\_\+background.\+frag} for the background fragment shader
\end{DoxyItemize}

\subsection*{Result + Next}





You should get something like this\+:  You can now proceed to the next tutorial\+: \mbox{\hyperlink{tutorial_superimpose_background}{how to add a background}}! 